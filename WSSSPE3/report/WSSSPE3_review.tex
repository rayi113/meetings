\documentclass[11pt, oneside]{amsart}
\pdfoutput=1

\usepackage{amsmath}
\usepackage{amssymb}

\usepackage{color}
\usepackage{dcolumn}
\usepackage{float}
\usepackage{graphicx}
\usepackage[utf8]{inputenc}
\usepackage[T1]{fontenc}
\usepackage{lmodern}
\usepackage{multirow}
\usepackage{rotating}
\usepackage{subfigure}
\usepackage{psfrag}
\usepackage{tabularx}
\usepackage[hyphens]{url}
\usepackage{wrapfig}
\usepackage{longtable}
\usepackage{verbatim}
\usepackage{booktabs,multicol}

% The following three lines are used for displaying footnote in tables.
\usepackage{footnote}
\makesavenoteenv{tabular}
\makesavenoteenv{table}


\usepackage{enumitem}
\setlist{leftmargin=7mm}

%\setcounter{secnumdepth}{3}
%\setcounter{tocdepth}{3}


\usepackage[bookmarks, bookmarksopen, bookmarksnumbered]{hyperref}
\usepackage[all]{hypcap}
\urlstyle{rm}

\definecolor{orange}{rgb}{1.0,0.3,0.0}
\definecolor{violet}{rgb}{0.75,0,1}
\definecolor{darkgreen}{rgb}{0,0.6,0}
\definecolor{cyan}{rgb}{0.2,0.7,0.7}
\definecolor{blueish}{rgb}{0.2,0.2,0.8}
\definecolor{darkblue}{rgb}{0.1,0.1,0.9}

\newcommand{\todo}[1]{{\color{blue}$\blacksquare$~\textsf{[TODO: #1]}}}
\newcommand{\note}[1]{ {\textcolor{blueish}    { ***Note:      #1 }}}
\newcommand{\katznote}[1]{ {\textcolor{magenta}    { ***Dan:      #1 }}}
\newcommand{\clunenote}[1]{ {\textcolor{orange}    { ***Tom:      #1 }}}
\newcommand{\gabnote}[1]{ {\textcolor{cyan}    { ***Gabrielle:     #1 }}}
\newcommand{\nchnote}[1]{  {\textcolor{orange}      { ***Neil: #1 }}}
\newcommand{\manishnote}[1]{  {\textcolor{violet}     { ***Manish: #1 }}}
\newcommand{\davidnote}[1]{  {\textcolor{darkgreen}      { ***David: #1 }}}
\newcommand{\colinnote}[1]{ {\textcolor{red}    {***Colin: #1 }}}
\newcommand{\choinote}[1]{ {\textcolor{orange}    {***Choi: #1 }}}
\newcommand{\stevenote}[1]{ {\textcolor{darkblue}    {***Steve: #1 }}}

% Don't use tt font for urls
\urlstyle{rm}

% 15 characters / 2.5 cm => 100 characters / line
% Using 11 pt => 94 characters / line
\setlength{\paperwidth}{216 mm}
% 6 lines / 2.5 cm => 55 lines / page
% Using 11pt => 48 lines / pages
\setlength{\paperheight}{279 mm}
\usepackage[top=2.5cm, bottom=2.5cm, left=2.5cm, right=2.5cm]{geometry}
% You can use a baselinestretch of down to 0.9
\renewcommand{\baselinestretch}{0.96}

\sloppypar

\begin{document}

\title[]{Report on the Third Workshop on Sustainable Software for Science: Practice and Experiences (WSSSPE3)}

\author{Daniel S. Katz$^{(1)}$,
Kyle E.\ Niemeyer$^{(2)}$,
Sou-Cheng T. Choi$^{(3)}$,
James Hetherington$^{(4)}$,
Frank L\"{o}ffler$^{(5)}$,
Dan Gunter$^{(6)}$,
Ray Idaszak$^{(7)}$,
Steven R. Brandt$^{(5)}$,
Mark A. Miller$^{(8)}$,
Sandra Gesing$^{(9)}$,
Nick D. Jones$^{(10)}$,
Nic Weber$^{(11)}$,
Gabrielle Allen$^{(12)}$,
Colin C. Venters$^{(13)}$,
Ethan Davis$^{(14)}$,
Lorraine Hwang$^{(15)}$,
Ilian Todorov$^{(16)}$,
Abani Patra$^{(17)}$,
Miguel de Val-Borro$^{(18)}$,
}


%
\thanks{{}$^{(1)}$ Computation Institute, 
University of Chicago \& Argonne National Laboratory, Chicago, IL, USA; d.katz@ieee.org}
%
\thanks{{}$^{(2)}$ School of Mechanical, Industrial, and Manufacturing Engineering, 
Oregon State University, Corvallis, OR, USA; kyle.niemeyer@oregonstate.edu}
%
\thanks{{}$^{(3)}$ NORC at the University of Chicago and Illinois Institute of Technology, Chicago, IL, USA; sctchoi@uchicago.edu}
%
\thanks{{}$^{(4)}$ Research Software Development Group, University College London, UK; j.hetherington@ucl.ac.uk}
%
\thanks{{}$^{(5)}$ Center for Computation \& Technology, Louisiana State University, Baton Rouge, LA, USA; \{knarf,sbrandt\}@cct.lsu.edu}
%
\thanks{{}$^{(6)}$ Lawrence Berkeley National Laboratory, Berkeley, USA; dkgunter@lbl.gov}
%
\thanks{{}$^{(7)}$ RENCI at the University of North Carolina at Chapel Hill, Chapel Hill, NC, USA; rayi@renci.org}
%
\thanks{{}$^{(8)}$ University of California, San Diego, CA, USA; mmiller@sdsc.edu}
%
\thanks{{}$^{(9)}$ unit, institution, city, country; email}
%
\thanks{{}$^{(10)}$ New Zealand eScience Infrastructure (NeSI), University of Auckland, Auckland, NZ; nick.jones@nesi.org.nz}
%
\thanks{{}$^{(11)}$ unit, institution, city, country; email}
%
\thanks{{}$^{(12)}$ National Center for Supercomputing Applications, University of Illinois at Urbana-Champaign, Urbana, IL, USA; gdallen@illinois.edu}
%
\thanks{{}$^{(13)}$ School of Computing and Engineering, University of Huddersfield, Huddersfield, UK; C.Venters@hud.ac.uk}
%
\thanks{{}$^{(14)}$ unit, institution, city, country; email}
%
\thanks{{}$^{(15)}$  University of California, Davis, CA, USA; ljhwang@ucdavis.edu}
%
\thanks{{}$^{(16)}$ unit, institution, city, country; email}
%
\thanks{{}$^{(17)}$ unit, institution, city, country; email}
%
\thanks{{}$^{(18)}$ Department of Astrophysical Sciences, 
Princeton University, Princeton, NJ, USA; mdevalbo@astro.princeton.edu}
%
 

\begin{abstract}
This technical report records and discusses the Third Workshop on Sustainable
Software for Science: Practice and Experiences (WSSSPE3). 
%The workshop used an
%alternative submission and peer-review process, which led to a set of papers
%divided across five topic areas: 
The report includes a description of the keynote presentation and a set of lightning
talks, and focuses on the discussions, future steps, and future organization for
a set of self-organized working groups.
For each group, the report also includes a point of contact and a landing
page that can be used by those who want to join that group's future activities.
The main challenge left by the workshop is to see if the groups will carry
out these activities that they have scheduled, and how the WSSSPE community
can encourage its members to do so.
%%
%The workshop recognized that reliance on scientific software
%is pervasive in all areas of world-leading research today. The workshop
%participants then proceeded to explore different perspectives on the concept of
%sustainability. Key enablers and barriers of sustainable scientific software
%were identified from their experiences. In addition,
%recommendations with new requirements such as software credit files and software
%prize frameworks were outlined for improving practices in sustainable software
%engineering.
%%
%There was also broad consensus that formal
%training in software development or engineering was rare among the
%practitioners. Significant strides need to be made in building a sense of
%community via training in software and technical practices, on increasing their
%size and scope, and on better integrating them directly into graduate education
%programs.
%%
%Finally, journals can define and publish policies to improve reproducibility, whereas
%reviewers can insist that authors provide sufficient information and access to
%data and software to allow them reproduce the results in the paper. Hence a list of
%criteria is compiled for  journals to provide to reviewers so as to make it easier to
%review software submitted for publication as a ``Software Paper.''

\end{abstract}


\maketitle
%\newpage

%%%%%%%%%%%%%%%%%%%%%%%%%%%%%%%%%%%%%%%%%%%%%%%%%%%%%%%%%%%%
\section{Introduction} \label{sec:intro}
%%%%%%%%%%%%%%%%%%%%%%%%%%%%%%%%%%%%%%%%%%%%%%%%%%%%%%%%%%%%

%\katznote{example comment by Dan}
%
%\gabnote{example comment by Gabrielle}
%
%\nchnote{example comment by Neil}
%
%\manishnote{example comment by Manish}
%
%\davidnote{example comment by David}

The Third Workshop on Sustainable Software for Science: Practice and Experiences
(WSSSPE3)\footnote{\url{http://wssspe.researchcomputing.org.uk/wssspe3/}} was
held on 28--29 September 2015 in Boulder, Colorado, USA. Previous events in the
WSSSPE series are
WSSSPE1\footnote{\url{http://wssspe.researchcomputing.org.uk/wssspe1/}}~\cite{WSSSPE1-pre-report,WSSSPE1},
held in conjunction with SC13;
WSSSPE1.1\footnote{\url{http://wssspe.researchcomputing.org.uk/wssspe1-1/}}, a
focused workshop organized jointly with the SciPy
conference\footnote{\url{https://conference.scipy.org/scipy2014/participate/wssspe/}};
WSSSPE2\footnote{\url{http://wssspe.researchcomputing.org.uk/wssspe2/}}~\cite{WSSSPE2-pre-report,WSSSPE2},
held in conjunction with SC14; and
WSSSPE2.1\footnote{\url{http://wssspe.researchcomputing.org.uk/wssspe2-1/}}, a
focused workshop organized again jointly with
SciPy\footnote{\url{http://scipy2015.scipy.org/ehome/115969/286469/}}.

Progress in scientific research is dependent on the quality and accessibility of
software at all levels. Hence it is critical to address challenges related to
development, deployment, maintenance, and overall sustainability of reusable
software as well as education around software practices. These challenges can be
technological, policy based, organizational, and educational; and are of
interest to developers (the software community), users (science disciplines),
software-engineering researchers, and researchers studying the conduct of
science (science of team science, science of organizations, science of science
and innovation policy, and social science communities). The WSSSPE1 workshop
engaged a broad scientific community to identify challenges and best practices
in areas of interest to creating sustainable scientific software. WSSSPE2
invited the community to propose and discuss specific mechanisms to move towards
an imagined future for software development and usage in science and
engineering. But WSSSPE2 didn't have a good way to enact those mechanisms, or to
encourage the attendees to follow through on their intentions.

The WSSSPE3 workshop included multiple mechanisms for participation and
encouraged team building around solutions. It strongly encouraged participation
of early-career scientists, postdoctoral researchers, and graduate students,
with funds provided to the conference organizers by the Moore Foundation, the
National Science Foundation (NSF), and the Software Sustainability Institute, to
support the travel of potential participants who would not otherwise be able to
attend, early-stage researchers, and those from underrepresented groups. These
funds allowed 16 additional participants to attend.

This report is based on collaborative notes taken during the workshop, which
were linked from the GitHub issues that represented the potential and actual
working
groups\footnote{\url{https://github.com/issues?q=label\%3A\%22WSSSPE3+activity\%22}}.
Overall, the report discusses the organization work done before the workshop
(\S\ref{sec:preworkshop}); the keynote (\S\ref{sec:keynote}); a series of
lightning talks (\S\ref{sec:lightning}). The report also gives summaries of
action plans proposed by the working groups (\S\ref{sec:WGs}), and some
conclusions (\S\ref{sec:conclusions}). Lists of the organizing committee
(Appendix~\ref{sec:orgcom}), the registered attendees
(Appendix~\ref{sec:attendees}), and the travel award recipients
(Appendix~\ref{sec:awardees}) are compiled. Finally, the report includes longer
descriptions of the activities that occurred in each of the working groups that
made substantial progress
(Appendices~\ref{sec:appendix_best_practices}--\ref{sec:appendix_user_community}).

%%%%%%%%%%%%%%%%%%%%%%%%%%%%%%%%%%%%%%%%%%%%%%%%%%%%%%%%%%%%
\section{Calls for Participation} \label{sec:preworkshop}
%%%%%%%%%%%%%%%%%%%%%%%%%%%%%%%%%%%%%%%%%%%%%%%%%%%%%%%%%%%%

WSSSPE3 was based on the work done in WSSSPE1 and WSSSPE2, but aimed at starting
a process to make progress in sustainable software, as the calls for
participation said:

\begin{quote} The WSSSPE1 workshop engaged the broad scientific community to
identify challenges and best practices in areas relevant to sustainable
scientific software. WSSSPE2 invited the community to propose and discuss
specific mechanisms to move towards an imagined future practice of software
development and usage in science and engineering. WSSSPE3 will organize
self-directed teams that will collaborate prior to and during the workshop to
create vision documents, proposals, papers, and action plans that will help the
scientific software community produce software that is more sustainable,
including developing sustainable career paths for community members. These teams
are intended to lead into working groups that will be active after the workshop,
if appropriate, working collaboratively to achieve their goals, and seeking
funding to do so if needed. \end{quote}

The first call for participation requested lightning talks, where each author
could make a brief statement about work that either had been done or was needed,
with the goal of contributing to the discussion of one or more working groups.
There were~24 lightning talks submitted, and, after a peer-review process, 16 of
those were accepted, as discussed further in Section~\ref{sec:lightning}.

The first call also discussed the potential action topics that came out of
WSSSPE2, and requested additional suggestions. The combination of existing and
new topics led to the following 18 potential topics that were advertised in the
subsequent calls for participation:


\begin{quote}
\begin{itemize}
\renewcommand{\labelenumi}{\textbf{\theenumi}.}
\setlength{\rightmargin}{1em}

\item Development and Community
\begin{itemize}
\item Writing a white paper/review paper about best practices in developing
sustainable software
\item Documenting successful models for funding specialist expertise in software
collaborations
\item Creating and curating catalogs for software tools that aid sustainability
(perhaps categorized by domain, programming languages, architectures, and/or
functions, e.g., for code testing, documentation)
\item Documenting case studies for academia/industry interaction
\item Determining effective strategies for refactoring/improving legacy
scientific software
\item Determining principles for engineering design for sustainable software
\item Create a set of guidance giving examples of specific metrics for the
success of scientific software in use, why they are chosen, what they are
useful to measure, and any challenges/pitfalls; then publish this as a white
paper
\end{itemize}

\item Training
\begin{itemize}
\item Writing a white paper on training for developing sustainable software, and
coordinating multiple ongoing training-oriented projects
\item Developing curriculum for software sustainability, and ideas about where
such curriculum would be presented, such as a summer training institute
\end{itemize}

\item Credit
\begin{itemize}
\item Hacking the credit and citation ecosystem (making it work, or work better,
for software)
\item Developing a taxonomy of contributorship/guidelines for including software
contributions in tenure review
\item Documenting case studies of receiving credit for software contributions
\item Developing a system of awards and recognitions to encourage sustainable
software
\end{itemize}

\item Publishing
\begin{itemize}
\item Developing a categorization of journals that publish software papers
(building on existing work), and case studies of alternative publishing
mechanisms that have been shown to improve software discoverability/reuse, e.g.,
popular blogs/websites
\item Determining what journals that publish software paper should provide to
their reviewers (e.g., guidelines, mechanisms, metadata standards)
\end{itemize}

\item Reproducibility and Testing
\begin{itemize}
\item Building a toolkit that could allow conference organizers to easily add a
reproducibility track
\item Documenting best practices for code testing and code review
\end{itemize}

\item Documentation
\begin{itemize}
\item Develop landing pages on the WSSSPE website (or elsewhere) that enable the
community to easily find up-to-date information on a WSSSPE topic (e.g.,
software credit, scientific software metrics, testing scientific software).
\end{itemize}

\end{itemize}
\end{quote}

%%%%%%%%%%%%%%%%%%%%%%%%%%%%%%%%%%%%%%%%%%%%%%%%%%%%%%%%%%%%
\section{Keynote \label{sec:keynote}}
%\note{Lead: S.-C. Choi.  See \href{http://tinyurl.com/qbbqgsj}{slides} \& \href{http://tinyurl.com/q45kfcn}{video}.}}
%%%%%%%%%%%%%%%%%%%%%%%%%%%%%%%%%%%%%%%%%%%%%%%%%%%%%%%%%%%%

WSSSPE3 began with a keynote speech delivered by Professor Matthew Turk from the
Department of Astronomy, University of Illinois, titled \emph{Why Sustain
Scientific Software?}. Turk is a prolific scientific software practitioner and
has extensive experiences working on large collaborative projects employing
modern computing tools. He also co-organizes and champions WSSSPE events.

In his keynote address, Turk recapped the course of development of WSSSPE
workshops over the past few years, alongside his career development from a
postdoc to an academic. The first WSSSPE workshop was at the Supercomputing
conference (SC13) in 2013, but he observed that the notion of sustainable
scientific software drew in an audience beyond supercomputing. In the following
year, WSSSPE1.1 at SciPy had speakers talking about how software has been
sustained inside the scientific Python community. WSSSPE2 at SC14 had breakout
group discussions coming up with actionable items, and WSSSPE2.1 at SciPy 2015
was similar. Turk noted the different atmosphere of the surrounding large
conferences, despite similar WSSSPE participants.

WSSSPE3 left the traditional Supercomputing Conference environment this year,
and in Turk's words, this spoke to the fact that scientific software comes from
many different types of inquiries, deployment, strategies for maintenance,
users, and ways of measuring the value of a piece of software. It appeared to
Turk that the supercomputing community generally adopts some top-down
approaches, whereas the SciPy community more often than not uses more bottom-up
systems. The essential messages perceived were also often bipolar: the
supercomputing community thinks that software is getting harder, with exascale
computing and optimization issues in mind; but the SciPy community thinks that
software is becoming better, with emerging tools such as Jupiter and
productivity packages for research workflows. Admitting such comparisons are
somewhat unfair generalizations, Turk reminded the audience that the different
approaches bring different types of ideas to the table, and he welcomed WSSSPE3
being conducted outside existing preconceptions.

Returning to the topic of his talk, Turk invited the audience to picture
scientific software as a flower on a landscape under the Sun, which may
represent a number of measurable factors such as number of citations; growth of
a community and number of contributors; amount of funding; prestigious prizes
awarded; stability of the community in terms of leadership transitions, serving
community needs, not breaking test suites, and performance on new architectures.
But all these metrics are strictly speaking \emph{proxies} for the values and
the impact scientific software bears. What we can measure does not give us
direct insight---it just gives us proxies of insight.
  
Turk then moved onto various different definitions of sustainability. His
favorite one was ``keeping up with bug reports,'' where even if no new features
were added, the software remains sustainable. Another definition of
sustainability Turk mentioned was ``adding of new features,'' or ``maintaining the
software for a long period of time'' such as the cases of TeX or LaTeX with
community help. A notion Turk heard often at supercomputing conferences was that
sustainable software ``continues to work on new architectures.'' Yet another
metric was ``people continuing to be able to learn how to use and apply the
software.'' A funder Turk heard talked about sustainability as ``continuing to get
funded.'' Turk also recalled that Greg Wilson, among others, said in WSSSPE1.1
that his view of sustainable software was software that ``continued to give the
same results over time.'' A last measure of sustainability Turk presented was
``the ability to transition between different people developing and using a piece
of software.''

At WSSSPE1, several models were presented for ensuring sustainability. Turk
considered that a familiar one was a funded piece of software where an external
agency provided funds to a group who are not necessarily exclusively working on
and developing the software, keeps it going, and provides it to the scientific
community. The model of productized software, in which a piece of software has
grown to the point that research groups or people are willing to support it with
some amount of funding, for instance, a subscription to use cloud services that
deploy a piece of software, or purchase of a piece of software. A final model
Turk felt conflicted about is a volunteer model that is traditional
old-school---not modern-day open source---development.

Turk discussed whether productizing scientific software was synonymous with
being sustainable and self-sufficient. He thought it was not necessarily the
case and furthermore, it could lead to a divergence of interests between users
and developers.

Turk reminded the audience that the volunteer model means unpaid labor. On this
note, he recommended Ashe Dryden's blog post on the ethics of unpaid labor and
the open source software
community\footnote{http://www.ashedryden.com/blog/the-ethics-of-unpaid-labor-and-the-oss-community}.
Often times, a person funded to work full time on a scientific project can spend
a small amount of time for working on a piece of software necessary for that
project. However, researchers' abilities to participate in that volunteer
community are not always the same and may not always be aligned with their
research projects. From Turk's experience, we cannot always rely on unpaid labor
and volunteer time to sustain a piece of software---this came down to the
notions of the top-down and the bottom-up approaches, i.e., the funded versus
the grassroots. However, Turk pointed out that bottom-up, volunteer-driven
projects can be just as large-scale as a top-down software development project.

Turk said that sustaining scientific software really meant to him conducting
scientific inquiries, often by some specific software, and sustaining the people
we care about, our careers, and the future of our fields. According to Turk, we
all have an invested stake in sustaining scientific software. Hence, having
``sustained'' projects can suffocate new projects, so we need to make sure we
don't cause novel ideas and packages to suffer at the hands of the status quo.

Turk talked about possible reasons why we want to sustain scientific software:
devotion to science and interests in pursuing the next stage of research; fun
and creative thrill in writing codes and papers; usefulness with measurable
impacts, for example, LINPACK and HDF groups providing data storage to
satellites, which goes beyond usefulness to necessity. Lastly, Turk presented
his wish-list of questions to be answered in the future:
%
\begin{itemize} 

\item How do we ship a product on time when dealing with a mix of funding models
and motivations especially when we rely on volunteers?

\item How do we know when it is time to end some software and move on? For
example, should we stop sustaining Python and switch to Julia and Javascript?
 
\item How can productized software balance its future versus its past, or the
new needs of the customers versus the existing needs of the development
community?

\item How can we help avoid burnout and retain the joy in the communities?

\item How can we reduce systemic bias, which goes back to the blog post of
Dryden especially on how ethics of unpaid labor disproportionately affect
underrepresented communities?

\end{itemize}

%%%%%%%%%%%%%%%%%%%%%%%%%%%%%%%%%%%%%%%%%%%%%%%%%%%%%%%%%%%%
\section{Lightning Talks} \label{sec:lightning}
%%%%%%%%%%%%%%%%%%%%%%%%%%%%%%%%%%%%%%%%%%%%%%%%%%%%%%%%%%%%
\begin{comment}
\note{
\href{http://wssspe.researchcomputing.org.uk/wssspe3/agenda/}{Slides.}}
\end{comment}

The lightning talks were intended to give an opportunity for attendees to
quickly highlight an important issue or a potential solution.
%
\begin{enumerate}
\item \textbf{Benjamin Tovar and Douglas Thain: \textit{Freedom vs.\ Stability:
Facilitating Research Training While Supporting Scientific Research}} Benjamin
Tovar presented a case study of ``The Cooperative Computing Lab'' at the
University of Notre Dame, which is a small group of individuals whose main tasks
are collaborating with people that have large-scale computing problems,
operating various parallel computer systems, conducting computer science
research, and developing open source software. One of the main challenges they
face is finding a balance between flexibility/training and stability/quality.
Their current solution for ensuring the latter was to add a software engineer
(the presenter) to the existing team of faculty and students, who now also
serves as a ``spring'' between flexibility and stability.

\item \textbf{Birgit Penzenstadler, Colin Venters, Christoph Becker, Stefanie
Betz, Ruzanna Chitchyan, Let\'{i}cia Duboc, Steve Easterbrook, Guillermo
Rodriguez-Navas, and Norbert Seyff: \textit{Manifesting the Ghost of the Future:
Sustainability}} The concept of sustainability has become a topic of interest in
the field of computing, which is evidenced by the increase in the number of
events that focus on the topic. Nevertheless, it isn't well understood yet.
Birgit Penzenstadler argued that we often define sustainability too narrowly.
Instead, sustainability at its heart is a systemic concept and must be viewed
from a range of different dimensions including environmental, economic,
individual, social, and technical. She introduced the Karlskrona Manifesto on
Software Design~\cite{Becker:2014}, which distills knowledge from a broad range
of related work on the topic of sustainability into a set of (mis-)perceptions
and principles. The manifesto does not proclaim that there is an easy,
one-size-fits-all solution around the corner, but rather points out that
sustainability is a ``wicked problem'' and is often misunderstood. Due to these
misperceptions, even though sustainability's importance is increasingly
recognized, many software systems are unsustainable. Even more alarming is that
most software systems' broader impacts on sustainability are unknown. To change
this, the Karlskrona Manifesto proposes nine principles and commitments. These
commitments are not dogmatic laws, but rather commitments to rethink, to move
beyond the silo mentality, and to analyze in more depth. As such, they do not
restrict, but rather open up a space for discussion.

\item \textbf{Abani Patra, Hossein Aghakhani, Nikolay Simakov, Matthew Jones,
and Tevfik Kosar: \textit{Integrating New Functionality Using Smart Interfaces
to Improve Productivity of Legacy Tools}} Abani Patra presented an example of
how the community using Titan2D, a Geoflow Simulation Software, increased the
productivity of their tools by improving both code and data layout. The main
obstacles in this change were the non-existence of a common version control
system for the source code, coupled with multiple versions of the same code
base, the fixed format of input files, that many input values were set as
compilation flags, and that the internal data layout was not suitable for modern
technologies (vectorization, accelerators). The approach of the Titan2D
developers included reinforcing the code structure using multiple layers of
Python and C++ interfaces, and a redesign of the data layout to be more suitable
for modern CPUs and accelerators.

\item \textbf{Abigail Cabunoc Mayes, Bill Mills, Arliss Collins, and Kaitlin
Thaney: \textit{Collaborative Software Development as Sustainable Software:
Lessons from Open Source}} Abigail Cabunoc Mayes combines two properties of open
source that, together, create a suitable habitat for sustainable software. The
first of the two properties, public, does not only mean public code. It also
includes public discussions, a public process of including contributions, and an
open license. The second property, participatory, stresses the importance of
reaching out to the community and helping potential new members by providing
better documentation and learning experiences, like code review and examples of
good first bug reports. Together, Abigail concluded, these two properties not
only lead to higher quality, reusability, and ease of understanding, but also
eventually, to sustainability.

\item \textbf{Louise Kellogg and Lorraine Hwang: \textit{Advancing Earth Science
through Best Practices in Open Source Software: Computational Infrastructure for
Geodynamics}} Lorraine Hwang presented experiences with CIG: the Computational
Infrastructure for Geodynamics, a community software with a worldwide user base.
Like others, their main goals include high usability, sustainability, and
reproducibility. In order to achieve these goals, communication channels have
been developed, such as mailing lists, wikis, workshops, hackathons, tutorials,
and webinars. In order to contribute to the infrastructure, codes must adhere
to specified minimum standards with the desire that all codes are working toward target standards.
%, but ideals are also given to encourage reaching the aforementioned goals. 
These include, e.g., the use of version control,
certain coding styles, the presence and nature of code tests and documentation,
and certain user workflows.

\item \textbf{Lorraine Hwang, Joe Dumit, Alison Fish, Louise Kellogg, Mackenzie
Smith, and Laura Soito: \textit{Software Attribution for Geoscience Applications
in the Computational Infrastructure for Geodynamics}} In a second talk, Lorraine
Hwang mentioned various ways to cite efforts within the framework, including
science papers, code papers, user manuals, and the CIG website. An analysis of
the resulting citations showed that $80\%$ of papers that use CIG codes mention
the code name, and about the same number includes a citation. Only about $20\%$
acknowledge CIG. Within the same sample of papers, about one fifth use an URL to
cite codes (including non-CIG codes), and only about one eighth specify the
version used. Compared to other codes, CIG seems to be much better cited. 
In part, this is attributed to the fact that CIG requires that donated software provide a citable paper specified in the User Manual. 
The project is working on tools and methods to generate attribution information automatically.
%This,
%in part, is attributed to suggested citation forms communicated to the
%community, and runtime tools that help collecting a relevant citation list
%automatically.

\item \textbf{Mike Hildreth, Jarek Nabrzyski, Da Huo, Peter Ivie, Haiyan Meng,
Douglas Thain, and Charles Vardeman: \textit{Data And Software Preservation for
Open Science (DASPOS)}} DASPOS is an NSF-funded multi-disciplinary effort,
located at Notre Dame and Chicago, that links the high energy physics effort to
biology, astrophysics, digital curation, and other disciplines. It includes
physicists, digital librarians, as well as computer scientists, and aims to
achieve some commonality across disciplines. Examples are meta-data descriptions
of archived data, computational descriptions, descriptions of how data was
processed, questions such as whether computation replication can be automated,
and what the impact of access policies on the preservation infrastructure is.
One of the products of this effort is a suite of tools that deals with this
preservation, and the questions was posed whether that software itself is
sustainable. Points that were brought up included the need for a user community
depending on a given software, and the need to provide added value for its
users. An important method to achieve this was to work with the user community
from the start, and to budget that way. For the specific example of the
preservation software, this means that besides adding value to the community, it
needs to be transparent to their workflows, i.e., not requiring additional
effort to preserve ``research objects''.

\item \textbf{James Hetherington, Jonathan Cooper, Robert Haines, Simon
Hettrick, James Spencer, Mark Stillwell, Mike Croucher, Christopher Woods, and
Susheel Varma: \textit{An update from UK Research Software Engineers}} James
Hetherington started by listing some of the problems research software faces,
which include low levels of reuse and poor standard of verification. For a long
time, technical solutions to such technical problems were focus of the eResearch
community, including research software distributions, grids, middleware and
workflows. Some limited adoption can be seen today in research communities, but
the main problems have not been solved to a sufficient level. Hetherington
hypothesized that instead of technical solutions, social innovation is needed: a
new role in the academic system focused on research software that combines the
best parts of a craftsperson and a scholar. However, social innovation in
centuries-old institutions is hard. Alternatives to such a new role would have
to include rewards for good research software, recognition of software as
academic output, and rejecting submissions based on irreproducible computational
results. Some advantages of research software engineering (RSE) groups include
the possibility of training in reproducible computational research, providing
collaborations for researchers who don't want to be programmers, and synergies
with research computing platforms. The success of such a group would be measured
by the output and quality of the research software, and could be part of
Research Computing or faculty. Members would not be independent researchers, but
would have to have a research background. An attempt to form a community of RSE
groups within the UK has been underway for several years, including funding from
the UK Research Council. An open question is whether this approach can be
adopted in other countries, including the USA.

\item \textbf{Dan Gunter, Sarah Poon, and Lavanya Ramakrishnan: \textit{Bringing
the User into Building Sustainable Software for Science}} Dan Gunter's main
question was, ``What is needed to develop sustainable software?''. Beyond the
usual suspects of funding, great developers, good design, and software
engineering practices, Dan placed the users. He explained this using a
traditional software-development model starting from gathering requirements, and
reaching release through a design, development, and testing phase. The main
deficiency with this approach was pointed out to be the too-late interaction
with users. Instead, an alternative approach was proposed that, at first, skips
the development phase and repeatedly goes through requirement gathering, design
and user interaction/learning phases, and only eventually starts development
once an agreement is reached, leading to an increased user satisfaction, higher
adoption, and eventually to sustained software.

\item \textbf{Dan Gunter, Adam Arkin, Rick Stevens, Robert Cottingham, and
Sergei Maslov: \textit{Challenges of a Sustainable Software Platform for
Predictive Biology: Lessons Learned on the KBase Project}} In this talk, Dan
Gunter presented experiences and lessons learned as part of KBase, an open
software and data platform for addressing the grand challenge of systems
biology: predicting and designing biological function. KBase is a unified system
that integrates data and analytical tools for comparative functional genomics of
microbes, plants, and their communities. However, it is also a collaborative
environment for sharing methods and results, and placing those results in the
context of knowledge in the field. Being a large, multi-institutional project,
one of the big challenges is to agree on standard to enable a single,
maintainable system. Working in isolation does not work (anymore) within this
field, and the community in the field also didn't have standards for software
engineering. This is contrasted to computer science research, where these
approaches shorten design cycles, leading to more time for highly rewarded
activities like publishing, performance studies, graduating students, or
protecting ideas before publication. Instead of this more
traditional approach, KBase uses a variation on the ``Scrum'' methodology.
After picking projects and team members, four to five teams work on projects for about two
weeks before a one-week evaluation by the executive committee, and after which new
teams and projects might be chosen to restart the ``agile'' development cycle.
This process is intentionally open and documented.

\item \textbf{Yolanda Gil, Chris Duffy, Chris Mattmann, Erin Robinson, and Karan
Venayagamoorthy: \textit{The Geoscience Paper of the Future Initiative: Training
Scientists in Best Practices of Software Sharing}} Erin Robinson presented an
approach to overcome some hurdles in scientific publishing: disconnects between
experimental data, research software, and publications. A current effort in the
Geosciences is the ``Geoscience Paper of the Future,'' which includes four
elements. First, it forms a modern paper including text, data, and pointers to
supplementary materials. Second, it is reproducible, including data processing,
workflow, and visualization tools. Third, it is part of open science, which
includes being publicly available under open licenses, and providing meta-data.
Last, but not least, it uses digital scholarship elements like persistent
identifiers for software, data, and authors, and it cites both data and
software. OntoSoft is a tool to help this effort, providing software stewardship
for the Geosciences. As part of this initiative, a special issue of a journal in
Geoscience areas is planned to include only Geoscience papers of the future,
with submissions open until the end of 2015. In addition, training sessions are
provided to geoscientists to learn best practices in software and data sharing,
provenance documentation, and scholarly publication.

\item \textbf{Neil Chue Hong: \textit{Building a Scientific Software
Accreditation Framework}} Neil Chue Hong presented a proposal to build a
scientific software accreditation framework. One of the aims of such a framework
would be to measure how ``good'' a given piece of software is, and to evaluate
how this can be effectively measured in the first place. This can be compared to
the effort of the standardized, easy to read, and understandable nutritional
labeling of food, which only contain a small set of categories. However, such a
framework for software would be more difficult due to different existing
community norms. The challenges such a framework faces include that many
measurements are subjective, that many metrics are too costly, and that
self-assessment needs to be encouraged. Possible categories would include
availability, usability, transferability, among others. Such a framework could
enable both improvement of specific software, as well as comparisons of similar
software. An accreditation by such a framework could then be part of software
management plans, ensuring that software is accessible and reusable throughout
the proposed project and beyond.

\item \textbf{Jeffrey Carver: \textit{On the Need for Software Engineering
Support for Sustainable Scientific Software}} Carver argued that for scientific
software to be truly sustainable, there is a need for developers to use
appropriate software engineering practices. His experience interacting with
scientific teams indicates that choosing and tailoring these practices is not a
trivial exercise. There is a general culture clash between software engineering
and science that hinders our ability to communicate and choose appropriate
methods. In addition, many experienced scientific software developers appear to
be unaware of software engineering practices that may be beneficial to them. The
most appropriate software engineering practices are those that are lightweight,
properly tailored, and focus on the key software development problems faced by
scientists. In order to increase the use of software engineering in science, we
need more documented success stories. These successes need to be socialized 
%\choinote{not sure if ``socialize'' can be used in such a way}
within the scientific community through workshops like the Software Engineering
for Science workshop
series\footnote{\href{http://www.SE4Science.org/workshops}{http://www.SE4Science.org/workshops}}
and the new Software Engineering track in \textit{Computing in Science and
Engineering} magazine.

\item \textbf{Matthias Bussonnier: \textit{User Data Collection in Open Source}}
This talk highlighted an attempt to solve the common problem for open source
development: it is difficult to collect information about how many people use
particular software, how often, which version, which parts of the software, or
on which operating system. Current solutions include surveys, but these have
high uncertainties. A different approach is based on automatic ``call-backs''
that collect these information at runtime and send it to a central place for
analysis. Problems with this approach include obtaining agreement from the user,
legal issues, increased maintenance (servers), ethical questions, and also the
lack of a common infrastructure. Some of these problems are of a social nature
and have to be solved as such, but the last problem (a missing common
infrastructure) is attacked by the sempervirens project~\cite{sempervirens},
which is developing common APIs and a library implementation for common,
repeating tasks such as obtaining user consent. Results are uploaded not
directly to project servers, but to neutral third parties that only publish
aggregated statistics to projects.

\item \textbf{Alice Allen: \textit{We're giving away the store! (Merchandise not
included)}} Alice Allen described the Astrophysics Source Code Library
(ascl.net), an increasingly used way to obtain a unique ID for astrophysics
software that is indexed by indexing services and can be cited. ASCL offers
clones of existing infrastructure, provides server space and computing
resources, shares innovations, and permits moves elsewhere. Users provide a
domain name, then control and configure their site and use it as intended,
gather their codes as they wish, share innovations, and protect the provided
computing environment.

\item \textbf{Stan Ahalt, Bruce Berriman, Maxine Brown, Jeffrey Carver, Neil
Chue Hong, Allison Fish, Ray Idaszak, Greg Newman, Dhabaleswar Panda, Abani
Patra, Elbridge Gerry Puckett, Chris Roland, Douglas Thain, Selcuk Uluagac, and
Bo Zhang: \textit{Scientific Software Success: Developing Metrics While
Developing Community}} The effort behind this talk by Ray Idaszak started from a
break-out group at an NSF SI2 workshop in 2015, and centers around creating a
framework for creation of metrics for scientific software. This framework would
improve both the metrics and the software it evaluates, and could also serve as
a tool for building a community around the idea. With especially the latter idea
(building a community) in mind, a software ``peer review group'' would be
created, representing stakeholders who will self-review software created by
their respective communities, and will concurrently develop metrics. The whole
project should be community-governed, without a single institution overseeing
the activities or infrastructure, with the hope to evolving community-generated
and adopted standards. The generation of metrics would be tied to the actual
evaluation of software, creating an incentive by improving the evaluated
software itself during this process. The framework code would provide
infrastructure for the creation of metrics and evaluation, and forums for
generation of software success metrics. It would also support code reviews of
the evaluated software. An open question is whether it is possible to fit the
resulting metrics in a common template. So far, this is still in a design phase,
with a white paper at the CSESSP workshop 2015 and this talk, but the WSSSPE
workshops are seen as a forum for the community to assemble and act, and is
planned to be used also in the future to build this community and framework.

\end{enumerate}

%%%%%%%%%%%%%%%%%%%%%%%%%%%%%%%%%%%%%%%%%%%%%%%%%%%%%%%%%%%%
\section{Working Groups} \label{sec:WGs}
%%%%%%%%%%%%%%%%%%%%%%%%%%%%%%%%%%%%%%%%%%%%%%%%%%%%%%%%%%%%

%\todo{go through subsections that have both content here and appendix, and add pointer from text in the subsection here to the corresponding appendix - done in line for Software Credit group already}

\input{wg_main_best_practices}
\input{wg_main_funding_spec_expert}
%\input{wg_main_catalogs}
\input{wg_main_industry_interaction}
\input{wg_main_legacy_SW}
\input{wg_main_eng_design}
\subsection{Useful Metrics for Scientific Software}

\subsubsection{Why it is important}
\todo{short text here}

\subsubsection{Fit with related activities}
\todo{short text here - can include links/cites}

\subsubsection{Discussion}
\todo{short-ish text here}

\subsubsection{Plans}
\todo{short text here - not bullets}

\subsubsection{Landing Page}
\todo{link to landing page}

\input{wg_main_training}
\input{wg_main_software_credit}
\input{wg_main_publishing_SW}
\input{wg_main_user_community}

%%%%%%%%%%%%%%%%%%%%%%%%%%%%%%%%%%%%%%%%%%%%%%%%%%%%%%%%%%%%
\section{Conclusions} \label{sec:conclusions}
%%%%%%%%%%%%%%%%%%%%%%%%%%%%%%%%%%%%%%%%%%%%%%%%%%%%%%%%%%%%

In WSSSPE3, we attempted to take what we learned from WSSSPE1 and WSSSPE2 in how
we can collaboratively build a workshop agenda and turn that into an ongoing
community activity. The success or failure of this effort will only
become apparent over time.

The workshop had two components, presentations and working groups. The
presentations, in the first half day of the workshop, included an inspirational
keynote and a set of lightning talks. We used lightning talks for two reasons:
first, the need of some participants to have a slot on the agenda to justify
their attendance; and second, as a way to get new ideas across to all the
attendees. We broke with the tradition of requiring the lightning talk submitters
to self-publish their papers, and instead used a common peer-review
platform\footnote{\url{http://easychair.org}}, choosing to publish their
slides on the workshop website instead.

The working groups met for a small part of the first half day and all of the
second day, with the exception of some short periods for the groups to report
back to the collected workshop attendees. Each group determined a set
of activities that the members could do to advance sustainable software in
a particular area.

The results of these group sessions made it clear that
there are many interlinked challenges in sustainable software, and that
while these challenges can be addressed, doing so is difficult because they
generally are not the full-time job of any of the attendees. 
As was the case in WSSSPE2 as well, the participants were willing
to dedicate their time to the groups while they were at the meeting, but
afterwards, they went back to their (paid) jobs. 

We need to determine how
to tie the WSSSPE breakout activities to people's jobs, so that they feel that
continuing them is a higher priority than it is now, perhaps through funding the
participants, or through funding coordinators for each activity, or perhaps by
getting the workshop participants to agree to a specific schedule of activities
during the workshop as we have tried to do in WSSSPE3.  It remains to be
seen, however, if the participants will meet the schedules they set.

The overall challenge
left to the sustainable software community is perhaps one of organization: how
to combine the small partial efforts of a large number of people to impact a much
larger number of people: those who develop and use scientific software.  While
WSSSPE might help focus the actions of the groups, something more is needed
to incentivize the wider community, which is a generalization of the sustainable
software problem itself.





%%%%%%%%%%%%%%%%%%%%%%%%%%%%%%%%%%%%%%%%%%%%%%%%%%%%%%%%%%%%
\section*{Acknowledgments} \label{sec:acks}
%%%%%%%%%%%%%%%%%%%%%%%%%%%%%%%%%%%%%%%%%%%%%%%%%%%%%%%%%%%%

Work by Katz was supported by the National Science Foundation while working at
the Foundation. Any opinion, finding, and conclusions or recommendations
expressed in this material are those of the author(s) and do not necessarily
reflect the views of the National Science Foundation. Choi's work is supported
in part by the National Science Foundation research grant DMS-1522687 and
a WSSSPE3 travel award. Hetherington was funded by
the Software Sustainability Institute, RCUK grants EP/H043160/1 and EP/N006410/1.

\todo{feel free to add stuff here}


\appendix
%%%%%%%%%%%%%%%%%%%%%%%%%%%%%%%%%%%%%%%%%%%%%%%%%%%%%%%%%%%%
\section{Organizing Committee}  \label{sec:orgcom}
%%%%%%%%%%%%%%%%%%%%%%%%%%%%%%%%%%%%%%%%%%%%%%%%%%%%%%%%%%%%
%\todo{Do we want email addresses here?}
The following is the list of organizers of WSSSPE3.

{\scriptsize
\begin{longtable}{lll}
\input{orgcom}
\end{longtable}
}
 

%%%%%%%%%%%%%%%%%%%%%%%%%%%%%%%%%%%%%%%%%%%%%%%%%%%%%%%%%%%%
\section{Attendees}  \label{sec:attendees}
%%%%%%%%%%%%%%%%%%%%%%%%%%%%%%%%%%%%%%%%%%%%%%%%%%%%%%%%%%%%
%\todo{Do we want email addresses here?}
The following is a list of participants registered for the WSSSPE3 workshop.

{\scriptsize
\begin{longtable}{lll}
Alice Allen & \href{mailto:aallen@ascl.net}{aallen@ascl.net}&Astrophysics Source Code Library\\
Gabrielle Allen & \href{mailto:gdallen@illinois.edu}{gdallen@illinois.edu}&NCSA\\
Janine Aquino & \href{mailto:janine@ucar.edu}{janine@ucar.edu}&UCAR/NCAR Earth Observing Laboratory\\
Steven Brandt & \href{mailto:sbrandt@cct.lsu.edu}{sbrandt@cct.lsu.edu}&Louisiana State University\\
Jed Brown & \href{mailto:jed@jedbrown.org}{jed@jedbrown.org}&CU Boulder\\
Matthias Bussonnier & \href{mailto:bussonniermatthias@gmail.com}{bussonniermatthias@gmail.com}&UC Berkeley\\
Jeffrey Carver & \href{mailto:carver@cs.ua.edu}{carver@cs.ua.edu}&University of Alabama\\
Emily Chen & \href{mailto:echen35@illinois.edu}{echen35@illinois.edu}&NCSA\\
Sou Cheng Choi & \href{mailto:sctchoi@uchicago.edu}{sctchoi@uchicago.edu}&NORC at UChicago / IIT\\
Nancy Collins & \href{mailto:nancy@ucar.edu}{nancy@ucar.edu}&NCAR\\
Ethan Davis & \href{mailto:edavis@ucar.edu}{edavis@ucar.edu}&UCAR Unidata\\
Davide DelVento & \href{mailto:ddvento@ucar.edu}{ddvento@ucar.edu}&NCAR/CISL\\
Yuhan Ding & \href{mailto:yding2@hawk.iit.edu}{yding2@hawk.iit.edu}&Illinois Institute of Technology\\
Tim Dunne & \href{mailto:tim.dunne@knowinnovation.com}{tim.dunne@knowinnovation.com}&KnowInnovation \\
Ward Fisher & \href{mailto:wfisher@ucar.edu}{wfisher@ucar.edu}&UCAR/Unidata\\
Sandra Gesing & \href{mailto:sandra.gesing@nd.edu}{sandra.gesing@nd.edu}&University of Notre Dame\\
Josh Greenberg & \href{mailto:greenberg@sloan.org}{greenberg@sloan.org}&Sloan Foundation\\
Dan Gunter & \href{mailto:dkgunter@lbl.gov}{dkgunter@lbl.gov}&LBNL\\
Ted Habermann & \href{mailto:thabermann@hdfgroup.org}{thabermann@hdfgroup.org}&The HDF Group\\
James Hetherington & \href{mailto:jamespjh@gmail.com}{jamespjh@gmail.com}&University College London\\
Neil Chue Hong & \href{mailto:N.ChueHong@software.ac.uk}{N.ChueHong@software.ac.uk}&Software Sustainability Institute\\
Elisabeth Huffer & \href{mailto:beth@lingualogica.net}{beth@lingualogica.net}&Lingua Logica/NASA \\
Lorraine Hwang & \href{mailto:ljhwang@ucdavis.edu}{ljhwang@ucdavis.edu}&UC Davis - CIG\\
Raymond Idaszak & \href{mailto:rayi@renci.org}{rayi@renci.org}&RENCI; University of North Carolina at Chapel Hill\\
Elizabeth Jessup & \href{mailto:jessup@cs.colorado.edu}{jessup@cs.colorado.edu}&University of Colorado Boulder\\
Nick Jones & \href{mailto:nick.jones@nesi.org.nz}{nick.jones@nesi.org.nz}&New Zealand eScience Infrastructure (NeSI)\\
Daniel Katz & \href{mailto:dsk@uchicago.edu}{dsk@uchicago.edu}&U Chicago \& Argonne\\
Iain Larmour & \href{mailto:Iain.Larmour@epsrc.ac.uk}{Iain.Larmour@epsrc.ac.uk}&EPSRC (UK)\\
Frank Löffler & \href{mailto:knarf@cct.lsu.edu}{knarf@cct.lsu.edu}&Louisiana State University\\
Suresh Marru & \href{mailto:smarru@iu.edu}{smarru@iu.edu}&Indiana University\\
Ryan May & \href{mailto:rmay@ucar.edu}{rmay@ucar.edu}&UCAR/Unidata\\
Abigail Cabunoc Mayes & \href{mailto:abigail@mozillafoundation.org}{abigail@mozillafoundation.org}&Mozilla Foundation\\
Jeff McWhirter & \href{mailto:jeff.mcwhirter@gmail.com}{jeff.mcwhirter@gmail.com}&Geode Systems\\
Constantinos Michailidis & \href{mailto:costa.michailidis@knowinnovation.com}{costa.michailidis@knowinnovation.com}&Knowinnovation\\
Don Middleton & \href{mailto:don@ucar.edu}{don@ucar.edu}&NCAR\\
Mark Miller & \href{mailto:mark.alan.miller2@gmail.com}{mark.alan.miller2@gmail.com}&SDSC\\
Pate Motter & \href{mailto:pate.motter@colorado.edu}{pate.motter@colorado.edu}&University of Colorado\\
Jaroslaw Nabrzyski & \href{mailto:naber@nd.edu}{naber@nd.edu}&University of Notre Dame\\
Patrick Nichols & \href{mailto:pnichols@ucar.edu}{pnichols@ucar.edu}&National Center for Atmospheric Research\\
Kyle Niemeyer & \href{mailto:Kyle.Niemeyer@oregonstate.edu}{Kyle.Niemeyer@oregonstate.edu}&Oregon State University\\
Laura Owen & \href{mailto:lowen@illinois.edu}{lowen@illinois.edu}&NCSA\\
Abani Patra & \href{mailto:abani@buffalo.edu}{abani@buffalo.edu}&Univ at Buffalo\\
Grace Peng & \href{mailto:grace@ucar.edu}{grace@ucar.edu}&National Center for Atmospheric Research\\
Birgit Penzenstadler & \href{mailto:Birgit.Penzenstadler@csulb.edu}{Birgit.Penzenstadler@csulb.edu}&California State University Long Beach\\
Lindsay Powers & \href{mailto:lpowers@hdfgroup.org}{lpowers@hdfgroup.org}&The HDF Group\\
Bernie Randles & \href{mailto:randles@ucla.edu}{randles@ucla.edu}&UCLA\\
Erin Robinson & \href{mailto:erinrobinson@esipfed.org}{erinrobinson@esipfed.org}&Foundation for Earth Science\\
Daniel Sellars & \href{mailto:dan.sellars@canarie.ca}{dan.sellars@canarie.ca}&CANARIE Inc\\
Nikolay Simakov & \href{mailto:nikolays@buffalo.edu}{nikolays@buffalo.edu}&SUNY University at Buffalo\\
Ian Taylor & \href{mailto:ian.j.taylor@gmail.com}{ian.j.taylor@gmail.com}&Cardiff University\\
Ilian Todorov & \href{mailto:ilian.todorov@stfc.ac.uk}{ilian.todorov@stfc.ac.uk}&Science \& Technology Facilities Council, UK\\
Benjamin Tovar & \href{mailto:btovar@nd.edu}{btovar@nd.edu}&University of Notre Dame\\
Gregory Tucker & \href{mailto:gtucker@colorado.edu}{gtucker@colorado.edu}&University of Colorado at Boulder\\
Matthew Turk & \href{mailto:mjturk@illinois.edu}{mjturk@illinois.edu}&NCSA\\
Colin Venters & \href{mailto:colin.venters@googlemail.com}{colin.venters@googlemail.com}&University of Huddersfield\\
Alexander Vyushkov & \href{mailto:avyushko@nd.edu}{avyushko@nd.edu}&University of Notre Dame\\
Fraser Watson & \href{mailto:fwatson@nso.edu}{fwatson@nso.edu}&National Solar Observatory\\
Nic Weber & \href{mailto:nmweber@uw.edu}{nmweber@uw.edu}&University of Washington\\
Daniel Ziskin & \href{mailto:ziskin@ucar.edu}{ziskin@ucar.edu}&NCAR - ACOM\\

\end{longtable}
}

%%%%%%%%%%%%%%%%%%%%%%%%%%%%%%%%%%%%%%%%%%%%%%%%%%%%%%%%%%%%
\section{Travel Award Recipients}  \label{sec:awardees}
%%%%%%%%%%%%%%%%%%%%%%%%%%%%%%%%%%%%%%%%%%%%%%%%%%%%%%%%%%%%
%\todo{Do we want email addresses here?}
The following is the list of travel award recipients for the WSSSPE3 workshop.

{\scriptsize
\begin{longtable}{lll}
\input{awardees}
\end{longtable}
}


\input{wg_appendix_best_practices}
\input{wg_appendix_funding_spec_expert}
%\input{wg_appendix_catalogs}
\input{wg_appendix_industry_interaction}
%\input{wg_appendix_legacy_SW}
\input{wg_appendix_eng_design}
%%%%%%%%%%%%%%%%%%%%%%%%%%%%%%%%%%%%%%%%%%%%%%%%%%%%%%%%%%%%
\section{Metrics Working Group Discussion}
\label{sec:appendix_metrics}
%%%%%%%%%%%%%%%%%%%%%%%%%%%%%%%%%%%%%%%%%%%%%%%%%%%%%%%%%%%%

Gabrielle Allen \footnote{email: \href{mailto:gdallen@illinois.edu}{gdallen@illinois.edu}} will serve as the point of contact for this working group.


%%%%%%%%%%%%%%%%%%%%%%%%%%%%%%%%%%%%%%%%%%%%%%%%%%%%%%%%%%%%
\subsection{Group Members}
%%%%%%%%%%%%%%%%%%%%%%%%%%%%%%%%%%%%%%%%%%%%%%%%%%%%%%%%%%%%

\begin{itemize}
\item Gabrielle Allen -- National Center for Supercomputing Applications
\item Emily Chen -- University of Illinois at Urbana-Champaign
\item Neil Chue Hong -- U.K. Software Sustainability Institute
\item Ray Idaszak -- RENCI, University of North Carolina at Chapel Hill
\item Iain Larmou -- Engineering and Physical Sciences Research Council
\item Bernie Randles -- University of California, Los Angeles
\item Dan Sellars -- Canarie
\item Fraser Watson -- National Solar Observatory
\end{itemize}

\subsection{Summary of Discussion}





\subsection{Description of Opportunity, Challenges, and Obstacles}






\subsection{Key Next Steps}






\subsection{Plan for Future Organization}






\subsection{What Else is Needed?}






\subsection{Key Milestones and Responsible Parties}






\subsection{Description of Funding Needed}





\input{wg_appendix_training}
\input{wg_appendix_software_credit}
\input{wg_appendix_publishing_SW}
\input{wg_appendix_user_community}

\bibliographystyle{vancouver}

\bibliography{wssspe}
\end{document}

